\documentclass[a4paper]{article}

\usepackage[T1]{fontenc}
\usepackage[utf8]{inputenc}
\usepackage{lmodern}
\usepackage[french]{babel}
\usepackage{enumitem}

\usepackage{graphicx}
\usepackage{caption}

\title{Pré-rapport du projet de Gygès}
\author{Valentin \textsc{Lemière} - Guillaume \textsc{Desquesnes}}
\date{}

\begin{document}
\maketitle

\section{Présentation}

	Pour ce projet, il nous a été demandé de créer une intelligence artificielle
	pour le jeu de Gygès.

	\subsection{Le jeu de Gygès}
		\paragraph{Présentation} Le Gygès est un jeu de stratégie combinatoire abstrait, c'est à dire un jeu
		à deux joueurs jouant à tour de rôle, où tous les éléments du jeu sont connus et où le hasard n'intervient
		pas. Ce jeu se joue sur un plateau 6x6, plus deux bases, avec des pions de valeurs 1 à 3 (représentés par
		des anneaux sur le pions) correspondant à leur capacité de déplacement. Les pions peuvent se déplacer en ligne
		ou en colonne mais pas en diagonale. Les pions n'appartiennent à aucun joueur, chaque joueur ne pouvant
		déplacer que les pions se trouvant sur la ligne non vide la plus proche de sa base.

		\begin{figure}[h]
			\centering
			\includegraphics[width=0.5\textwidth]{Gyges.png}
			\caption{Exemple de plateau de Gygès}
			\label{fig:plateau_de_gyges}
		\end{figure}


		\paragraph{Le but} Le but du jeu est de réussir à capturer la base adverse avec un pion,
		le premier joueur réussissant cet objectif gagne et met fin à la partie.

		\paragraph{Les règles} Le Gygès possède un certain nombre de règles qui le démarque de jeux plus
		\og{}standard\fg{}:
			\begin{itemize}
				\item Un pion se déplace d'autant de case que ça valeur, ni plus ni moins ;

				\begin{figure}[ht]
					\centering
					\begin{minipage}[b]{0.28\linewidth}
					\centering
					\includegraphics[width=\textwidth]{move1.png}
					\caption{Déplacement d'un pion 1}
					\label{fig:figure1}
					\end{minipage}
					\hspace{0.5cm}
					\begin{minipage}[b]{0.28\linewidth}
					\centering
					\includegraphics[width=\textwidth]{move2.png}
					\caption{Déplacement d'un pion 2}
					\label{fig:figure2}
					\end{minipage}
					\hspace{0.5cm}
					\begin{minipage}[b]{0.28\linewidth}
					\centering
					\includegraphics[width=\textwidth]{move3.png}
					\caption{Déplacement d'un pion 3}
					\label{fig:figure3}
					\end{minipage}
					\caption{Déplacement des pions}
				\end{figure}

				\item Un pion peut se déplacer sur les lignes et les colones, mais pas les diagonales ;
				\item Un pion ne peut pas passer par une case occupée ;
				\item Si la dernière case du mouvement d'un pion est libre alors il se pose dessus et le tour
				du joueur s'achève ;
				\item Sinon le joueur a deux possibilités :
					\begin{description}
						\item[Rebondir] Le pion rebondit sur le pion d'arrivée d'autant de case que
							ce dernier a d'anneaux ;
							\begin{figure}[h]
								\centering
								\includegraphics[width=0.25\textwidth]{rebond.png}
								\caption{Exemple de rebond}
								\label{fig:plateau_de_gyges}
							\end{figure}
						\item[Remplacer] Le pion remplace le pion d'arrivée, ce dernier pourra alors être
							reposé sur n'importe quelle case libre du plateau à l'exception des lignes avant
							la première ligne de l'adversaire.
							\begin{figure}[h]
								\centering
								\includegraphics[width=0.6\textwidth]{remplacement.png}
								\caption{Exemple de remplacement}
								\label{fig:plateau_de_gyges}
							\end{figure}
					\end{description}
				\item Un pion ne peut passer sur une ligne que dans un sens ;
				\item Un pion ne peut pas passer par une base, il ne peut que y aboutir ;
				\item Pour gagner un pion doit finir exactement sur la base adverse, il ne doit pas lui
					rester de mouvement.
			\end{itemize}

		\paragraph{Déroulement d'une partie} Le joueur sud pose ses pions sur sa première ligne
		suivi du joueur nord. Chaque joueur déplace ensuite à tour de rôle, en commencant par le
		joueur sud, un pion de leur ligne. La partie s'arrête lorsqu'un joueur a capturé la base
		adverse.

		\begin{figure}[ht]
			\centering
			\begin{minipage}[b]{0.4\linewidth}
			\centering
			\includegraphics[width=\textwidth]{ex1.png}
			\end{minipage}
			\hspace{0.5cm}
			\begin{minipage}[b]{0.4\linewidth}
			\centering
			\includegraphics[width=\textwidth]{ex2.png}
			\end{minipage}

			\begin{minipage}[b]{0.4\linewidth}
			\centering
			\includegraphics[width=\textwidth]{ex3.png}
			\end{minipage}
			\hspace{0.5cm}
			\begin{minipage}[b]{0.4\linewidth}
			\centering
			\includegraphics[width=\textwidth]{ex4.png}
			\end{minipage}
			\caption{Example de déroulement d'une partie de Gygès}
		\end{figure}


	Nous développons ce projet pour Grégory \textsc{Bonnet}. Il pourra être utilisé
	par des joueurs souhaitant s'entrainer ou jouer seul.

	\subsection{Objectifs}
		Nous avons comme objectifs:

		\vspace{1em}
		\begin{itemize}
			\item De commencer par étudier le jeu de Gygès pour pouvoir le modéliser:
				\begin{itemize}
					\item lister les règles spéciales ;
					\item coder le mouvement des pions ;
					\item déterminer quels pions chaque joueur peut déplacer.
				\end{itemize}

			\vspace{1em}
			\item Ensuite de créer un module de jeu permettant le déroulement d'une partie:
				\begin{itemize}
					\item création d'un plateau ;
					\item initialisation des IA ;
					\item déroulement tour à tour jusqu'a la victoire d'un des deux joueurs.
				\end{itemize}

			\vspace{1em}
			\item lister les coups possible à partir d'un plateau:
				pour chaque pions déplaçable lister quelles sont les positions attégniables ;

			\vspace{1em}
			\item déterminer si un coup est légal: existe-t-il un mouvement permettant
				de passer du plateau de départ à celui d'arrivé ;

			\vspace{1em}
			\item créer une fonction d'évaluation de plateau ;

			\vspace{1em}
			\item implémenter un algorithme de recherche alpha-beta, permettant de ne pas
				jouer au hasard ;

			\vspace{1em}
			\item trouver et implémenter des fonctions d'évaluation plus performantes,
				pour jouer de meilleurs coups ;

			\vspace{1em}
			\item trouver et implémenter des algorithmes de recherche plus performants,
				permettant une recherche plus profonde et donc de meilleurs coups ;

			\vspace{1em}
			\item développer une interface graphique pour pouvoir jouer facilement contre
				une IA.
		\end{itemize}

\section{Le point de départ}

	\subsection{Logiciels}
		\`A ce jour, nous n'avons trouvé qu'une seule application permettant de jouer
		contre une intelligence artificielle au Gygès, on peut la trouver sur le site suivant: \\
		\texttt{www.gyges.com}.
		Cependant, cette application n'est pas capable de trouver tout les coups. De plus
		cette application n'ayant pas son code source disponible, il nous est impossible de l'utiliser pour ajouter et
		tester différents algorithmes, dans le but de trouver le plus adapté pour résoudre les
		problèmes du Gygès.


		Nous n'avons donc aucun composant sur lequel baser notre projet, il va falloir tout développer.

	\subsection{Algorithmes}
		Il existe aujourd'hui divers algorithmes permettant de trouver un bon coup dans un jeu,
		il va donc falloir déterminer quel sera le plus adapté. Voici les algorithmes qui nous semblent les
		plus pertinants, d'après nos recherches, pour notre problème:

		\paragraph{Minimax} Un des plus simples est le \emph{minimax}, cet algorithme assigne à un n\oe{}ud
		la valeur minimale ou maximale de ses n\oe{}uds fils, respectivement lorsque le n\oe{}ud resprésente
		le tour de l'adversaire ou le tour du joueur. Pour un n\oe{}ud feuille sa valeur correspond à la valeur
		retournée par la fonction d'évaluation pour le plateau représenté par ce n\oe{}ud. Cet algorithme cherche donc
		à minimiser les pertes tout en maximisant les gains.

		\paragraph{Search EXtension} Une amélioration du \emph{minimax} est le \emph{Search EXtension},
		cet algorithme permet d'avoir une profondeur de recherche variable. Plus le coup semble intéressant, plus
		la branche sera explorée et au contraire, moins le mouvement est intéressant, moins on explorera la branche.

		\paragraph{\'Elagage Alpha-Beta} Une optimisation du \emph{minimax}, l'élagage \emph{Alpha-Beta}
		permet de réduire le nombre de n\oe{}uds évalués. L'algorithme évite d'évaluer des n\oe{}uds dont
		on est sûr que leur qualité sera inférieure à un n\oe{}ud déjà évalué, il permet donc d'optimiser
		grandement l'algorithme minimax sans en modifier le résultat.

		\paragraph{Negamax} Une variante du \emph{minimax}, en se basant sur le fait
		que $max(a,b) = -min(-a,-b)$ et que la valeur de coup pour le joueur adverse et la négation du
		coup pour le joueur courant, pour simplifier l'algorithme en n'utilisant que des \emph{max}
		au lieu d'une alternance \emph{max}, \emph{min}.

		\paragraph{MTD-f} Une optimisation du \emph{minimax}, le \emph{MTD-f} se sert de tables de transposition
		pour garder en mémoire la valeur d'un plateau et ainsi ne noter qu'une seule fois les plateaux identiques.
		C'est une des variantes du \emph{minimax} la plus rapide.

		\paragraph{MCTS} La recherche arborescente Monte-Carlo est une recherche où l'on construit un arbre de recherche
		n\oe{}uds à n\oe{}uds en fonction du résultat de parties simulées aléatoirement.

\section{Le point d'arrivée espéré}
	\subsection{Ce que nous allons réaliser}
		Nous allons réaliser un logiciel doté d'une interface graphique permettant à un ou deux joueurs
		de jouer au jeu de Gygès. L'interface comportera un menu permettant de sélectioner l'intelligence
		artificielle que le joueur veut affronter, pour jouer il suffira de cliquer sur le pion à déplacer
		puis de cliquer sur sa destination. De plus pour chaque intelligence artificielle, il sera possible
		de régler la profondeur de recherche ou le temps maximal alloué pour permettre de doser la difficulté.


		\vspace{1em}
		Enfin il sera possible de générer des parties IA contre IA sans interface graphique
		afin de pouvoir avoir accès à des données de jeu qui sont facilement exploitables.

	\subsection{Ce que nous allons développer}
		Nous allons développer un moteur de jeu, celui ci permettra de générer les coups légaux à partir d'un plateau
		et de vérifier si un coup est légal, c'est à dire qu'il est possible de passer du plateau de départ à celui
		d'arrivé.


		Nous définirons aussi plusieurs fonctions d'évaluation de plateaux, permettant ainsi d'avoir plusieurs
		comportements possible pour l'intelligence artificielle et/ou plusieurs vitesses de calcul.

		Enfin, nous implémenterons plusieurs algorithmes de recherche permettant de déterminer le meilleur coup à jouer
		d'après notre intelligence artificielle pour une profondeur	de recherche donnée, afin de pouvoir trouver lesquels
		se prettent le plus aux problèmes que pose le Gygès: très fort facteur de branchement et règles atypiques.

		\vspace{1em}
		Nous allons, dans un premier temps, développer les algorithmes Negamax et MTD-f, avec un élagage Alpha-Beta ainsi
		qu'un perceptron multicouche comme l'une des fonctions d'évaluations.

	\subsection{Comment cela répond-il aux objectifs ?}
		Ce produit permettra le déroulement d'une partie, quelque soit le nombre de joueur humain.
		Il pourra lister l'ensemble des coups possible à partir d'un plateau et vérifier
		si un coup est légal.

		\vspace{1em}
		De plus, il contiendra plusieurs fonctions d'évaluation aux caractéristiques variées, performantes sur
		le temps ou sur la qualité de l'évaluation. De même, il possèdera plusieurs algorithmes de recherches
		variés performants d'un point de vue temporel
		ou d'utilisation de la mémoire.

		\vspace{1em}
		Enfin, il possèdera une interface graphique qui permet de jouer facilement au jeu.

	\subsection{Ce que nous avons déjà développé}
		Nous avons déjà développé le moteur de jeu ainsi qu'un système générique permettant d'implémenter
		aisément différents types d'algorithme de recherche, ainsi qu'un système similaire pour les
		fonctions d'évaluations.

		\vspace{1em}
		Nous n'avons implémenté qu'un seul algorithme de recherche pour le moment :
		un algorithme \emph{negamax} avec élagage alpha-beta.

		\vspace{1em}
		Nous avons développé plusieurs fonctions d'évalution se basant sur la position des pions sur le plateau,
		indépendament les uns des autres ou non. Nous avons ainsi pu déterminer que trouver une fonction d'évaluation
		pertinante est très compliqué. C'est pour cela que nous allons implémenter un perceptron multicouche, qui est un
		algorithme permettant de régresser une fonction inconnue, ici celle d'évaluation du plateau de Gygès, grace à un
		apprentissage à base d'analyse de parties déjà jouées.

\section{Organisation du travail}
	\subsection{Techniques utilisées}
		Nous utilisons une technique de développement agile: \emph{l'extreme programming}. Cette méthode se
		base sur des cycles de développement rapides, ce qui nous permet d'avoir de fréquentes versions
		fonctionnelles et donc d'avoir des retours régulier du client sur le projet et de pouvoir ainsi
		s'adapter facilement à ses nouveaux besoins.

		\vspace{1em}
		De plus cela s'adapte bien à notre programation en binôme.

	\subsection{Répartition du travail}
		D'un point de vue recherche, chacun recherche différents algorithmes permettant d'accèlerer le temps
		de recherche alpha-beta, ainsi que différentes fonctions d'évaluation de plateaux.


		Du point de vue de l'implémentation, nous travaillons majoritairement ensemble, sur un seul ordinateur.
		Cela nous permet de corriger et de trouver plus rapidement les bugs, ainsi que d'optimiser et harmoniser
		le code.


		De plus cela facilite aussi la compréhension globale du code afin que chacun puisse aisément implémenter
		de nouvelles fonctionnalité	de manière séparée.


		De même nous nous efforçons de rendre le code le plus modulaire possible afin de pouvoir facilement rajouter
		de nouvelles fonctions d'évaluation ou de nouveaux algorithmes de recherches.

	\subsection{Problèmes}
		Plusieurs problèmes se posent, tout d'abord il est difficile de trouver une fonction d'évaluation pertinante
		pour noter un plateau selon un joueur, le Gygès étant asymètrique, qui ne soit pas trop complexe car cette
		fonction sera appelée un très grand nombres de fois.


		Ensuite il nous faut implémenter des algorithmes de parcours, cependant ces algorithmes d'apparence simple deviennent
		complexe lorsqu'il s'agit de les implémenter. De plus il est difficile de déterminer si un algorithme est adapté
		pour résoudre le Gygès, il faut donc le tester pour savoir si il est adapté ou non.


		Enfin il faudra optimiser le code et les algorithmes au maximum de nos capacités, afin de reduire la complexité et la mémoire
		utilisée, pour soit reduire le temps de calcul ou augmenter la profondeur de recherche.

\section{Conclusion}
	En conclusion, ce projet est avant tout un projet de recherche visant à trouver et tester différents algorithmes de recherche
	de coup ainsi que des fonctions d'évaluations sur le jeu de Gygès. Il nous faudra effectuer de nombreux tests pour classer
	les algorithmes selon leur temps de calul et les fonctions d'évaluation selon leur performances et leur capacité à faire gagner le joueur.
	L'extreme programming est un modèle de développement qui s'adapte bien à la nature itérative de notre projet.

\newpage
\appendix
\section{Lexique}
	\begin{description}[style=nextline]
		\item[Algorithme Search EXtension] Une amélioration du \emph{minimax} est le \emph{Search EXtension},
		cet algorithme permet d'avoir une profondeur de recherche variable. Plus le coup semble intéressant, plus
		la branche sera explorée et au contraire, moins le mouvement est intéressant, moins on explorera la branche.

		\item[\'Elagage Alpha-Beta] Une amélioration du \emph{minimax} est le \emph{Search EXtension},
		cet algorithme permet d'avoir une profondeur de recherche variable. Plus le coup semble intéressant, plus
		la branche sera explorée et au contraire, moins le mouvement est intéressant, moins on explorera la branche.

		\item[Intelligence artificielle] Système informatique mimant les capacités de réflexions
		d'un être humain. Ici fait référence à la partie du programme émulant un joueur.

		\item[Facteur de branchement] Nombre de plateaux possible depuis un plateau départ.

		\item[Fonction d'évaluation] Fonction prenant en entrée un plateau de jeu et sortant la valeur de
		ce plateau.

		\item[Horizon] Limite de calcul dans l'arbre de recherche. Fait référence au fait qu'un très bon coup peut
		se cacher derrière le dernier coup calculé, mais qu'il est impossible de le savoir.

		\item[Jeu combinatoire abstrait] Type de jeu où deux joueurs ou équipes d'affrontent et jouent à tour de rôle,
		tous les éléments sont connus et où le hasard n'intervient pas.

		\item[Minimax] Simple, cet algorithme assigne à un n\oe{}ud la valeur minimale ou maximale de ses
		n\oe{}uds fils, respectivement lorsque le n\oe{}ud resprésente le tour de l'adversaire ou le tour
		du joueur. Pour un n\oe{}ud feuille sa valeur correspond à la valeur retournée par la fonction
		d'évaluation pour le plateau représenté par ce n\oe{}ud. Cet algorithme cherche donc à minimiser
		les pertes tout en maximisant les gains.

		\item[MTD-f] Une optimisation du \emph{minimax}, le \emph{MTD-f} se sert de tables de transposition
		pour garder en mémoire la valeur d'un plateau et ainsi ne noter qu'une seule fois les plateaux identiques.
		C'est une des variantes du \emph{minimax} la plus rapide.

		\item[Negamax] Une variante du \emph{minimax}, en se basant sur le fait
		que $max(a,b) = -min(-a,-b)$ et que la valeur de coup pour le joueur adverse et la négation du
		coup pour le joueur courant, pour simplifier l'algorithme en n'utilisant que des \emph{max}
		au lieu d'une alternance \emph{max}, \emph{min}.

		\item[Perceptron multicouche] Réseau de neurones formel organisé en plusieurs couche utilisant une fonction de sortie
		de type sigmoïde. Utiliser comme classifieur non linéaire ou pour la régression de fonction inconnue.

		\item[Recherche Monte-Carlo (Monte-Carlo Tree Search)] La recherche arborescente Monte-Carlo est une recherche
		où l'on construit un arbre de recherche n\oe{}uds à n\oe{}uds en fonction du résultat de parties simulées aléatoirement.

		\item[Table de transposition] Permet de conserver en mémoire l'évaluation d'un n\oe{}ud et ainsi éviter une réévalutation
		coûteuse.
	\end{description}

\section{Sources}
	Toutes les images de plateau de Gygès proviennent du site: \texttt{www.gyges.com}.

\section{Tests}
	Nous avons effectué un test comparant le temps d'exécution d'un coup, suivant les fonctions d'évaluation, en fonction de la
	profondeur de recherche.

	\begin{figure}[h]
		\centering
		\includegraphics[width=\textwidth]{compaFonction.png}
		\caption{Comparaison des fonctions d'évaluations}
		\label{fig:plateau_de_gyges}
	\end{figure}

	On remarque donc que la seconde fonction voit son temps requis augmenter exponentiellement avec la profondeur, elle
	est donc inadaptée à notre projet de Gygès car il nous faut une fonction d'évaluation rapide pour pouvoir tester
	un très grand nombre de plateaux.

\end{document}
