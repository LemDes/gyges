\section{Méthodologie de recherche}

	\paragraph{Valeur d'un coup} Pour établir des fonctions heuristiques nous avons du déterminer
	une manière d'évaluer un plateau, c'est à dire une façon de noter un coup afin de pouvoir déterminer
	un meilleur coup parmis l'ensemble des coups possibles. Pour cela nous avons essayé diverses stratégies contre
	des intelligences artificielles déjà existantes ainsi que contre d'autres joueurs plus ou moins expérimentés.
	Nous avons ensuite gardé celles qui étaient efficace dans la plupart des cas.

	\paragraph{Plateau symètrique} Puisque le plateau est symètrique, nous avons décidé de noter uniquement les plateaux
	appartenant du joueur Nord. Ainsi pour évaluer les plateaux du joueur Sud nous effectuons une rotation du plateau.\\

	Soit $P$ le plateau initial:

	\[
		P = \left(
			\begin{array}{ c c c c }
				1 & 2 & 1 & 2\\
				0 & 4 & 0 & 0\\
				0 & 0 & 4 & 0\\
				2 & 1 & 1 & 2\\
			\end{array}
			\right)
			+
			\left(
			1, 0
			\right)
	\]

	\newpage
	alors $Q$ sera sa rotation:

	\[
		Q = \left(
			\begin{array}{ c c c c }
				2 & 1 & 1 & 2\\
				0 & 0 & 4 & 0\\
				0 & 4 & 0 & 0\\
				1 & 2 & 1 & 2\\
			\end{array}
			\right)
			+
			\left(
			0, 1
			\right)
	\]

	\paragraph{Mouvements asymètriques} Le fait que le jeu soit asymètrique offre plusieurs façon d'évaluer les plateaux.
	Il est ainsi possible d'évaluer uniquement sur les points positifs (resp. négatifs) du joueur ou bien,
	de prendre la différence des points positifs (resp négatifs) du joueur et de son adversaire. Dans le cas
	de la différence, ce calcul est possible car \[valeurJ_{1} \neq -1*valeurJ_{2}\] Cependant, cela implique une double
	évaluation du plateau qui augmente donc le temps de calcul et diminue la profondeur de recherche en un temps fixe.
