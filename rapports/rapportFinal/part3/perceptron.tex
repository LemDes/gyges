\section{Perceptron}

	%\subsection{Contexte}
		% Trouver une fonction d'évaluation correcte est difficile, l'alternative présenté
		% ici propose d'en approximer une par apprentissage via un perceptron.
		Trouver une fonction d'évaluation correcte est difficile, il est cependant possible
		d'en approximer une par apprentissage via un perceptron.

	\subsection{Le perceptron multicouche}
		Le perceptron multicouche est composé d'un réseau de neurones, reliés ensemble par des liens pondérés,
		organisé en plusieurs couches, d'une fonction d'activation non linéaire, telle que la tangente hyperbolique, et d'un
		algorithme de rétropropagation du gradient de l'erreur pour permettre de résoudre des problèmes non-linéaires.

		\begin{figure}[h]
			\centering
			\includegraphics[width=0.6\textwidth]{images/Perceptron_4layers.png}
			\caption{Exemple d'un perceptron multicouche}
			\label{fig:perceptron_multi}
		\end{figure}

		%\vspace{1em}
		%Voir figure \ref{fig:perceptron_multi} pour un exemple de perceptron multicouche.

	\subsection{Abandon}
		L'approximation d'une fonction par un perceptron multicouche nécessite de fournir des données
		déjà traitées. Tout d'abord, il est difficile de déterminer les plateaux intéressant et étant
		donné la quantité de plateau a traité. En effet il serait impossible de le faire sans fonction
		d'évaluation au vu de la limitation de temps du projet. Nous avons donc décidé d'abandonner
		l'implémentation du perceptron.
