\subsection{Maxpath eval}

	Cette heuristique est la plus complexe des trois, puisqu'elle note un plateau en
	fonction du nombre de chemin disponible. Cela implique donc de pouvoir calculer
	tout les chemins possible à partir des pions de notre ligne et de celle de l'adversaire.

	%Posséder un réseau de chemins fort,
	Posséder plusieurs chemin permettant d'accèder à plusieurs endroits similaire est une des
	clefs de la réussite au Gygès. Si vous construisez deux chemins distincts menant à la base adverse
	alors la victoire est assurée.

	\begin{figure}[h!]
	\[
		\begin{array}{r c l}

			score(C) &= & \sum\limits_{i=0}^{5} \sum\limits_{j=0}^{5}
			\left\{
			\begin{array}{r l}
				i^{C_{i,j}} & $si $ C_{i,j} \neq 0 \\
				0 & $sinon$ \\
			\end{array}
			\right. \\

			&& \\

			base(P) &= &
			\left\{
			\begin{array}{r l}
				10000 & $si la base sud de P est occupée$ \\
				0 & $sinon$ \\
			\end{array}
			\right. \\

			&& \\

			value(P) &= & base(P) + \sum\limits_{m \in M} score(C_m) \\

			&& \\

			maxpathEval(P,Q) &=& value(P) - value(Q)\\

		\end{array}
	\]
	\caption[Formule de maxpath eval]{Formule de maxpath eval avec $M$ l'ensemble des pions déplacables pour le joueur,
	$C_m$ l'ensemble des chemins atteignables à partir du pion $m$,
	$P$ la matrice représentant le plateau et
	$Q$ la matrice représentant le plateau retourné.}
	\label{fig:formula_maxpath_eval}
	\end{figure}
