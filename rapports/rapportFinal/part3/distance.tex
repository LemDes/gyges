\subsection{Distance eval}

	Le but de cette heuristique est de faire avancer les pions le plus possible afin de se rapprocher de
	la	base adverse pour créer des opportunités. Il est en effet évident que plus les pions sont proche
	de la base adverse, et par conséquent loin de la notre, plus les chances de victoires sont grandes.


	\begin{figure}[h!]
	\[
		\begin{array}{r c l}
			f(x,i) &=&
			\left\{
			\begin{array}{r l}
				i & $si$~ x \neq 0 \\
				0 & $sinon$\\
			\end{array}
			\right. \\

			&& \\

			base(P) &= &
			\left\{
			\begin{array}{r l}
				10000 & $si la base sud de P est occupée$ \\
				0 & $sinon$ \\
			\end{array}
			\right. \\

			&& \\

			value(P) &=&  base(P) + \sum\limits_{i=0}^{5} \sum\limits_{j=0}^{5} f(P_{i,j},i) \\

			&& \\

			distanceEval(P,Q) &=& value(P) - value(Q)\\
		\end{array}
	\]
	\caption[Formule de distance eval]{Formule de distance eval avec $P$ la matrice représentant le
	plateau et $Q$ la matrice représentant le plateau retourné.}
	\label{fig:formula_dist_eval}
	\end{figure}

	%\vspace{1em}
	%Voir figure \ref{fig:formula_dist_eval} pour une formule de cette heuristique.
