\subsection{Basic eval}

	Cette heuristique a pour but d'éviter d'avoir les pièces $n \in \{1,2,3\}$ sur la $n^{e}$ ligne du joueur tout
	en essayant de les placer sur la $n^{e}$ ligne adverse. En effet, si un pion de taille 2 est sur la deuxième
	ligne d'un joueur, ce joueur risquera de perdre dès lors que le pion précédent sera accessible à son adversaire. \\

	On ne pénalisera pas les pièces $n \in \{2,3\}$ sur la $(n-1)^{e}$ ligne puisqu'elles sont plus difficile
	d'accès et ont moins de chemin possible vers la base adverse. De plus cette version de l'heuristique fonctionne %Vérifier le truc des moins de chemin.
	très mal à faible profondeur, puisque la plupart des coups ont la même valeur. \\

	Enfin, comme toute les autres heuristiques, elle favorise fortement la victoire tout en pénalisant la défaite.

	\begin{figure}[h!]
	\[
		\begin{array}{r c l}
			f(x,i) &= &
			\left\{
			\begin{array}{r l}
				1 & $si$~ x = i + 1 \\
				0 & $sinon$\\
			\end{array}
			\right. \\

			&& \\

			base(P) &= &
			\left\{
			\begin{array}{r l}
				10000 & $si la base sud de P est occupée$ \\
				0 & $sinon$ \\
			\end{array}
			\right. \\

			&& \\

			value(P) &=&  base(P) + \sum\limits_{i=0}^{2} \sum\limits_{j=0}^{5} f(P_{i,j},i) \\

			&& \\

			basicEval(P,Q) &=& value(P) - value(Q)\\
		\end{array}
	\]
	\caption[Formule de basic eval]{Formule de basic eval avec $P$ la matrice
	représentant le plateau et $Q$ la matrice représentant le plateau retourné.}
	\label{fig:formula_basic_eval}
	\end{figure}

	%\vspace{1em}
	%Voir figure \ref{fig:formula_basic_eval} pour une formule de cette heuristique.
