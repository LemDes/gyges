\section{MTD-f}

    \subsection{Présentation}
        
        Le MTD-f est une amélioration des algorithmes de types minimax 
        permettant de réduire le temps de calcul d'un coup mais augmente
        l'utilisation totale de mémoire.

		\vspace{1em}
		
        L'évaluation de plusieurs milliers, voire millions de plateaux,
        selon la profondeur, rend le calcul d'un coup très long. Or nous
        avons besoin de réduire au maximum le temps passé à évaluer les
        plateaux.
        
        \vspace{1em}
        
        Cependant en évaluant l'ensemble des plateaux, nous allons évaluer plusieurs
        fois les mêmes plateaux, ce qui est une perte de temps. 
        Nous pourrions donc utiliser des tables de transposition pour garder
        en mémoire la valeur d'un plateau et ainsi ne pas le réévaluer.
        

        %~ \begin{figure}[h!]
%~ 
            %~ \begin{framed}
            %~ \begin{alltt}
%~ fonction AlphaBetaMémoire(\(racine\), \(\beta{}-1\), \(\beta{}\), \(d\))
    %~ \(entree\) = \(tableDeTransposition\).recupere(\(racine\))
    %~ si \(entrée\) existe  et  \(entrée.profondeur >= d \) alors
        %~ si \( entrée.valeurExacte  \) alors
            %~ retourne \(racine, entrée.valeur\)
        %~ si  \( entrée.elagéAlpha\)  alors
            %~ \(alpha=entrée.valeur\)
        %~ sinon si \( entrée.elagéBeta \) alors
            %~ \(beta=entrée.valeur\)
        %~ si \( \alpha{} >= \beta{} \) alors
            %~ retourne \(racine, entrée.valeur\)
    %~ si \( d = 0\) ou \(racine.estFeuille \)
        %~ \(valeur\)=evaluer(\(racine\))
        %~ si  \( valeur <= \alpha{} \) 
            %~ \(tableDeTransposition\).stocke(\(racine,elagéAlpha,d\))
        %~ sinon si \( valeur >= \beta{} \) 
            %~ \(tableDeTransposition\).stocke(\(racine,elagéBeta,d\))
        %~ si \( valeur <= \alpha{} \)
            %~ \(tableDeTransposition\).stocke(\(racine,valeurExacte,d\))
        %~ retourner \(racine, valeur\)
    %~ \(meilleur\)=-\infty{} \)
    %~ pour chaque \(coup\) suivant de \(racine\) faire
        %~ \(valeur\)=\(-1*\)AlphaBetaMémoire(\(coup,-1*\beta{} ,-1*\alpha{},d-1\)).\(valeur\)
        %~ si \( valeur > meilleur \)
            %~ \(meilleur=valeur\)
            %~ \(meilleurPlateau=coup \)
        %~ si \( meilleur > \alpha{} \)
            %~ \( \alpha{}=meilleur \)
        %~ si \( meilleur => \beta{} \)
            %~ sortir
    %~ si  \( meilleur <= \alpha{} \) 
        %~ \(tableDeTransposition\).stocke(\(meilleurPlateau,elagéAlpha,d\))
    %~ sinon si \( meilleur >= \beta{} \) 
        %~ \(tableDeTransposition\).stocke(\(meilleurPlateau,elagéBeta,d\))
    %~ si \( meilleur <= \alpha{} \)
        %~ \(tableDeTransposition\).stocke(\(meilleurPlateau,valeurExacte,d\))
    %~ retourner \(racine, valeur\)        
%~ fin fonction\end{alltt}\end{framed}
            %~ \caption{Pseudo-code de l'algorithme MTD-f}
            %~ \label{fig:pseudocode_mtdf}
        %~ \end{figure}
        


    %~ \subsection{Tables de transposition}
        
        \paragraph{Tables de transposition}
        Les tables de transpostion correspondent à des tableaux associatifs
        reliant un couple (plateau, profondeur) à l'évaluation du plateau.
        
        L'avantage des tableaux associatifs est que l'ajout et la lecture 
        d'éléments est en $O(1)$

    \subsection{Algorithme}
    
    Le pseudo-code de la fonction AlphaBetaAvecMémoire est disponible en
    anglais sur le site \texttt{http://people.csail.mit.edu/plaat/mtdf.html}
	
	\vspace{1em}
	Voir figure \ref{fig:pseudocode_mtdf} pour le pseudo-code du MTD-f.

    \begin{figure}[h!]

        \begin{framed}
        \begin{alltt}
fonction MTDF(\(racine\), \(f\), \(d\))
    \(g = f\)
    \(limiteMax = +\infty{}\)
    \(limiteMin = -\infty{}\)

    tant que \(limiteMin < limiteMax\) alors
        si \(g = limiteMin\) alors
            \(\beta{} = g+1\)
        sinon
           \(\beta{} = g\)
        fin si

        \(g =\) AlphaBetaAvecMémoire(\(racine\), \(\beta{}-1\), \(\beta{}\), \(d\))

        si \(g < \beta{}\) alors
            \(limiteMax = g\)
        sinon
            \(limiteMin = g\)
        fin si
    fin tant que

    retourner \(g\)
fin fonction

        \end{alltt}
        \end{framed}

        \caption{Pseudo-code de l'algorithme MTD-f}
    \label{fig:pseudocode_mtdf}
    \end{figure}


    \paragraph{Différences} Le minimax est l'algorithme le 
    plus simple pour résoudre le problème d'un arbre de recherche,
	il fonctionne en effectuant un parcours en profondeur de l'arbre.
	Il fut amélioré avec l'élagage alpha-beta notamment. 
    
    \vspace{1em}
    Le MTD-f est différent de l'élagage alpha-beta, en effectuant uniquement
	des recherche à fenêtre nulle cela lui permet d'élager bien plus et donc
	d'être plus rapide, mais cela ne lui permet pas d'avoir autant d'information,
	seulement une fenêtre sur le résultat. Pour avoir le résultat exact MTD-f appel
	donc un algorithme alpha-beta plusieurs fois jusqu'à trouver la bonne valeur.
	
	\vspace{1em}
	Le coup lié à la réexploration d'une partie de l'arbre de recherche, dû aux
	multiples appels à un algorithme alpha-beta, est compensé par l'utilisation
	de table de transposition évitant la réévaluation de n\oe{}uds déjà évalués.
	
	\vspace{1em}
	Une recherche à fenêtre nulle correspond à l'algorithmes alpha-beta
	où le paramètre alpha vaut beta-1.
