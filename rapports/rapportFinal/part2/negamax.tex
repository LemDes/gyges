\section{Concepts fondamentaux}

	\subsection{Arbre de recherche}
		La grande majorité des algorithmes de jeu
		représente une partie sous la forme d'un arbre de recherche, ce qui correspond
		plus ou moins à un arbre bicolore.
		La racine représente le plateau en cours, ses sous-n\oe{}uds, de couleur verte, les
		coups possibles du joueurs ; leurs sous-n\oe{}uds, de couleurs rouge, les coups de
		l'adversaire ; et ce jusqu'aux feuille de l'arbre. La taille de l'arbre correspond
		à la profondeur de recherche.

		\begin{figure}[h]
			\centering
			\includegraphics[width=0.40\textwidth]{images/arbre_recherche.png}
			\caption{Exemple d'arbre de recherche}
			\label{fig:arbre_recherche}
		\end{figure}

	\subsection{Minimax}
		Le minimax est un algorithme de recherche simple, il
		affecte une valeur à chaque noeud, pour une feuille cela représente le résultat
		de l'évaluation du plateau et pour les autres n\oe{}ud il faut distinguer deux cas:

		\begin{enumerate}
			\item le n\oe{}ud est vert, il représente un coup du joueur, alors sa valeur
				vaut la valeur minimal de ses sous-n\oe{}uds : minimiser le prochain coup adverse ; ou
			\item le n\oe{}ud est rouge, il représente un coup de l'adversaire, alors sa valeur
				vaut la valeur maximal de ses sous-n\oe{}uds : maximiser son prochain coup.
		\end{enumerate}

		La racine est de couleur rouge, elle représente le coup précédent de l'adversaire.

		\vspace{1em}
		On dit que le minimax consiste à minimiser la perte maximum, c'est à dire qu'il choisit
		un coup maximisant le gain du joueur tout en minimisant celui de l'adversaire.

		\paragraph{Exemple de l'utilisation du minimax}
			On commence par affecter des valeurs aux noeuds feuilles en utilisant une heuristique
			déterminant la qualité, ici, du plateau de jeu.

			\begin{figure}[h]
				\centering
				\includegraphics[width=0.4\textwidth]{images/minimax_ex_1.png}
				\caption{Minimax étape 1}
				\label{fig:minimax_ex_1}
			\end{figure}

			Ensuite on remonte leurs valeurs sur leur n\oe{}ud parent, ici un n\oe{}ud vert,
			on choisit donc la valeur minimale $valeur = min(sous noeuds)$.

			\begin{figure}[h]
				\centering
				\includegraphics[width=0.4\textwidth]{images/minimax_ex_2.png}
				\caption{Minimax étape 2}
				\label{fig:minimax_ex_2}
			\end{figure}

			Finalement on arrive à la racine, toujours rouge, ou l'on choisit la
			valeur maximale $valeur = max(sous_noeuds)$.

			\begin{figure}[h]
				\centering
				\includegraphics[width=0.4\textwidth]{images/minimax_ex_3.png}
				\caption{Minimax étape 3}
				\label{fig:minimax_ex_3}
			\end{figure}

			Nous avons donc déterminé qu'il faut jouer le $3^{\text{e}}$ coup, de valeur 7.

		\paragraph{Optimisation} Pour optimiser le parcours de l'arbre on effectuera un parcours
		en profondeur de l'arbre, affectant au fur et à mesure les valeurs des n\oe{}uds.

	\subsection{Negamax}
		Le negamax est une variante du minimax, qui se sert d'une propriété des jeux à
		somme nulle, pour en simplifier le code. Grâce au fait que maximiser $f$ équivaut à
		minimiser $-f$, principe du dualité, on a :

		\vspace{0.50em}
		\centerline{$max(a, b) = -min(-a, -b)$}
		\vspace{0.50em}

		\noindent
		on peut donc retirer le partie conditionnelle de l'algorithme et
		effectuer uniquement une fonction max sur l'opposé de la valeur,
		d'où le nom de negamax.


%	\subsection{Pseudo-code}
%
%		Voici un exemple de pseudo-code pour l'algorithme negamax:
%
%		\begin{figure}[h!]
%		\begin{mdframed}
%			\begin{alltt}
%fonction negamax(\(noeud\), \(profondeur\))
%    si \(noeud\) est une feuille ou si \(profondeur <= 0\) alors
%        retourner heuristique(\(noeud\)) \(*\) couleur du noeud
%    fin si
%
%    \(best = -\infty{}\)
%    pour chaque \(sous noeud\) de \(noeud\) faire
%        \(best = \)max(\(best\), \(-\)negamax(\(sous noeud\), \(profondeur-1\)))
%    fin pour chaque
%
%    retourner \(best\)
%fin fonction
%		\end{alltt}
%		\end{mdframed}
%
%	\caption{Pseudo-code de l'algorithme negamax}
%	\label{fig:pseudocode_negamax}
%	\end{figure}
