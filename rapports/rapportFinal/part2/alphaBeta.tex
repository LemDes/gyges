%\newpage
\subsection{\'Elagage alpha-beta}

	%\subsection{Contexte}
		Pour des jeux complexes tels que les échecs, le go ou le Gygès il
		n'est pas possible de représenter l'intégralité de l'arbre de recherche,
		on se limite donc à une profondeur donnée, mais ce n'est pas la seule
		technique possible pour réduire le temps de parcours de l'arbre de recherche.

		\vspace{1em}
		L'élagage alpha-beta est un algorithme visant à réduire le nombre de n\oe{}uds
		évalués par un algorithme minimax, pour cela il limite son exploration de l'arbre de
		recherche en n'évaluant pas les n\oe{}uds qui ne contribueront pas au calcul du gain de
		la racine. Plus exactement il ne va pas explorer plus profondement une branche dont
		le résultat sera forcement inférieur à un n\oe{}ud déjà évalué.
		
		\vspace{1em}
		L'algorithme alpha-beta tient son nom de ces deux paramètres alpha et beta
		représentant respectivement le minimum souhaité et le maximum espéré.
		Donc lorsqu'on tombe sur un coup ayant une valeur moins élevée (resp.
		plus élevée) que alpha (resp. beta) alors on sait que ce n\oe{}ud ne participe
		pas au calcul et donc peut être élagé.

	%\subsection{Exemple}
	%	%Repartons sur la figure \ref{fig:minimax_ex_3} représentant l'étape finale du minimax,
	%	%et regardons l'influence que l'élagage alpha-beta aurait eu.
	%
	%	%\begin{figure}[h]
	%	%	\centering
	%	%	\includegraphics[width=0.4\textwidth]{images/alphabeta_ex.png}
	%	%	\caption{Alpha-beta étape 1}
	%	%	\label{fig:aphaBeta_ex_1}
	%	%\end{figure}
	%
	%	Nous avons déjà sur le troisième n\oe{}ud vert une valeur de 4, ce qui est déjà
	%	supérieur aux autres n\oe{}uds vert, or le n\oe{}ud racine choisira le sous-n\oe{}ud
	%	avec la valeur maximal, donc quelque soit la valeur du second sous-n\oe{}ud rouge
	%	le n\oe{}ud choisis sera le troisième n\oe{}ud vert. Son évaluation est donc inutile,
	%	il se fait élager.
	%
	%	\vspace{1em}
	%	Ici seul l'évaluation d'un n\oe{}ud a été gagner, mais dans un arbre de recherche de très
	%	grande taille cela peut être des dizaines de n\oe{}uds ayant chacun de nombreux sous-n\oe{}uds
	%	dont l'évaluation aurait été gagner.

%	\subsection{Pseudo-code}
%		Voici un exemple de pseudo-code pour l'algorithme negamax avec élagage alpha-beta:
%
%		\begin{figure}[h!]
%		\begin{mdframed}
%			\begin{alltt}
%fonction negamax(node, depth, a, b, color)
%    if node is a terminal node or depth = 0
%        return color * the heuristic value of node
%    else
%        foreach child of node
%            val := -negamax(child, depth-1, -b, -a, -color)
%            {the following if statement constitutes alpha-beta pruning}
%            if val>=b
%                return val
%            if val>=a
%                a:=val
%        return a
%		\end{alltt}
%		\end{mdframed}
%
%	\caption{Pseudo-code de l'algorithme negamax}
%	\label{fig:pseudocode_negamax}
%	\end{figure}
