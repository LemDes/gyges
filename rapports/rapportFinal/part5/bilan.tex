\chapter{Conclusion}

	\section{Problèmes}
		Nous avons rencontré des difficultés à trouver des
		fonctions d'évaluation pertinentes, en effet le jeu
		de Gygès étant un jeu complexe que nous ne maîtrisons
		pas suffisament, il fut difficile de trouver une
		stratégie de jeu efficace et de la retranscrire en
		fonction mathématique.

	\section{Limites du projet}
		Nous n'avons pas eu le temps d'optimiser les différents algorithmes
		de recherche ni les fonctions d'évaluation, ce qui aurait permis
		d'augmenter la vitesse, ou la profondeur, de recherche.

		\vspace{1em}
		Aussi il n'est pas possible d'annuler un coup ou de voir
		comment l'IA a joué (chemin de déplacement du pion) à cause du
		temps limité du projet.

	\section{Perspectives}
		On pourrait envisager l'ajout de nouveaux algorithmes de
		recherche, non basé sur le minimax, et comparer leur
		performances avec le MTD-f.

		\vspace{1em}
		Il pourrait aussi être intéressant d'améliorer l'interface
		graphique pour pouvoir choisir les algorithmes et fonctions
		d'évaluation directement depuis l'interface. Voir aussi
		d'ajouter un mode multijoueur via internet.

		\vspace{1em}
		Finalement le rajout d'indications sur la qualité du coup
		que l'on vient de jouer serait un plus pour l'entrainement
		au Gygès, mais cela requiert la connaissance complète de
		l'arbre de recherche et est donc impossible.

	\section{Bilan}
		Ce projet fut l'occasion pour nous de réaliser un projet de a à z,
		comprenant un cahier des charges à prendre en compte ainsi que des
		limites de temps à respecter.

		%\vspace{1em}
		Cela a été l'opportunité de travailler en équipe et d'utiliser un
		gestionnaire de version.

		%\vspace{1em}
		Cela nous a aussi permis d'implémenter des algorithmes et concepts
		vu en cours et de mieux comprendre leur utilisation dans un projet
		de grande taille.

		\vspace{1em}
		Finalement ce projet nous a permis de mieux comprendre les principes et
		difficultés de la création d'une intelligence artificielle de jeu.
