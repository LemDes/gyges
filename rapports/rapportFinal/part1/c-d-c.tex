\section{Cahier des charges}

	Pour ce projet nous avions plusieurs étapes a réaliser, les étapes 1--4
	permettent d'avoir un module de jeu complet permetant le déroullement d'une partie,
	les étapes 5 et 6 de jouer de façon \og{} intelligente \fg{}, les étapes
	7 et 8 d'améliorer la performance de l'IA et finallement l'étape 9 de pouvoir
	facilement laisser un joueur humain affronter l'IA.

	\vspace{1em}
	\noindent
	Nous avions comme objectifs:

	\begin{enumerate}
		\item de commencer par étudier le jeu de Gygès pour pouvoir le modéliser ;
			\begin{enumerate}%[label=\Alph*]
				\item lister les règles spéciales ;
				\item coder le mouvement des pions ; et
				\item déterminer quels pions chaque joueur peut déplacer.
			\end{enumerate}

		\vspace{1em}
		\item de créer un module de jeu permettant le déroulement d'une partie ;
			\begin{enumerate}%[label=\Alph*]
				\item création d'un plateau ;
				\item initialisation des IA ; et
				\item déroulement tour à tour jusqu'a la victoire d'un des deux joueurs.
			\end{enumerate}

		\vspace{1em}
		\item de lister les coups possible à partir d'un plateau:
			pour chaque pions déplaçable lister quelles sont les positions attégniables ;

		\vspace{1em}
		\item de déterminer si un coup est légal: existe-t-il un mouvement permettant
			de passer du plateau de départ à celui d'arrivé ;

		\vspace{1em}
		\item de créer une fonction d'évaluation de plateau ;

		\vspace{1em}
		\item d'implémenter un algorithme de recherche alpha-beta, permettant de ne pas
			jouer au hasard ;

		\vspace{1em}
		\item de trouver et d'implémenter des fonctions d'évaluation plus performantes,
			pour jouer de meilleurs coups ;

		\vspace{1em}
		\item de trouver et implémenter des algorithmes de recherche plus performants,
			permettant une recherche plus profonde et donc de meilleurs coups ; et

		\vspace{1em}
		\item de développer une interface graphique pour pouvoir jouer facilement contre
			une IA.
	\end{enumerate}
