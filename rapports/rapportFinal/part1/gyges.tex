\section{Présentation}

	Le Gygès est un jeu de stratégie combinatoire abstrait, c'est à dire un jeu
	où deux joueurs jouent à tour de rôle, où tous les éléments du jeu sont connus et où le hasard n'intervient
	pas.

	\vspace{1em}
	Ce jeu se joue sur un plateau $6\times{}6$, plus deux bases, avec des pions de valeurs 1 à 3 (représentés par
	des anneaux sur le pions) correspondant à leur capacité de déplacement. Les pions peuvent se déplacer en ligne
	ou en colonne mais pas en diagonale.

	\vspace{1em}
	Les pions n'appartiennent à aucun joueur, chaque joueur ne pouvant
	déplacer que les pions se trouvant sur la ligne non vide la plus proche de sa base.

	\begin{figure}[h]
		\centering
		\includegraphics[width=0.35\textwidth]{images/Gyges.png}
		\caption{Exemple de plateau de Gygès}
		\label{fig:plateau_de_gyges}
	\end{figure}

	\paragraph{Le but} Le but du jeu est de réussir à capturer la base adverse avec un pion,
	le premier joueur réussissant cet objectif gagne et met fin à la partie.

	\paragraph{Les règles} Le Gygès possède un certain nombre de règles qui le démarque de jeux plus
	\og{}standard\fg{}.
		\begin{enumerate}
			\item Un pion se déplace d'autant de case que ça valeur, ni plus ni moins
			(voir figure 1.3--1.5) ;

			\begin{figure}[h!]
				\centering
				\begin{minipage}[h]{0.25\linewidth}
				\centering
				\includegraphics[width=\textwidth]{images/move1.png}
				\caption{Déplacement d'un pion 1}
				\label{fig:figure1}
				\end{minipage}
				\hspace{0.5cm}
				\begin{minipage}[h]{0.25\linewidth}
				\centering
				\includegraphics[width=\textwidth]{images/move2.png}
				\caption{Déplacement d'un pion 2}
				\label{fig:figure2}
				\end{minipage}
				\hspace{0.5cm}
				\begin{minipage}[h]{0.25\linewidth}
				\centering
				\includegraphics[width=\textwidth]{images/move3.png}
				\caption{Déplacement d'un pion 3}
				\label{fig:figure3}
				\end{minipage}
				\vspace{-1em}
			\end{figure}

			\item Un pion peut se déplacer sur les lignes et les colones, mais pas les diagonales ;
			\item Un pion ne peut pas passer par une case occupée ;
			\item Si la dernière case du mouvement d'un pion est libre alors il se pose dessus et le tour
			du joueur s'achève ;

			\item Sinon le joueur a deux possibilités :
				\begin{enumerate}
					\item rebondir: Le pion rebondit sur le pion d'arrivée d'autant de case que
						ce dernier a d'anneaux ; ou
						\begin{figure}[h!]
							\centering
							\includegraphics[width=0.29\textwidth]{images/rebond.png}
							\caption{Exemple de rebond}
							\label{fig:rebond}
						\end{figure}
					\item remplacer: Le pion remplace le pion d'arrivée, ce dernier pourra alors être
						reposé sur n'importe quelle case libre du plateau à l'exception des lignes avant
						la première ligne de l'adversaire.
						\begin{figure}[h!]
							\centering
							\includegraphics[width=0.59\textwidth]{images/remplacement.png}
							\caption{Exemple de remplacement}
							\label{fig:remplacement}
						\end{figure}
				\end{enumerate}
			\item Un pion ne peut passer sur une ligne que dans un sens ;
			\item Un pion ne peut pas passer par une base, il ne peut que y aboutir ; et
			\item Pour gagner un pion doit finir exactement sur la base adverse, il ne doit pas lui
				rester de mouvement.
		\end{enumerate}

	\paragraph{Déroulement d'une partie} Le joueur sud pose ses pions sur sa première ligne
	suivi du joueur nord. Chaque joueur déplace ensuite à tour de rôle, en commencant par le
	joueur sud, un pion de leur ligne. La partie s'arrête lorsqu'un joueur a capturé la base
	adverse.

	\begin{figure}[h!]
		\centering
		\begin{minipage}[b]{0.4\linewidth}
		\centering
		\includegraphics[width=\textwidth]{images/ex1.png}
		\end{minipage}
		\hspace{0.5cm}
		\begin{minipage}[b]{0.4\linewidth}
		\centering
		\includegraphics[width=\textwidth]{images/ex2.png}
		\end{minipage}

		\begin{minipage}[b]{0.4\linewidth}
		\centering
		\includegraphics[width=\textwidth]{images/ex3.png}
		\end{minipage}
		\hspace{0.5cm}
		\begin{minipage}[b]{0.4\linewidth}
		\centering
		\includegraphics[width=\textwidth]{images/ex4.png}
		\end{minipage}
		\caption{Example de déroulement d'une partie de Gygès}
	\end{figure}

	\vspace{1em}
	Les parties sont composées de mouvements complexes mais ne durent généralement
	pas plus d'une dizaines de coups.
