\section{Difficultées}

	Traditionnellement les techniques d'intelligence artificielle pour les jeux listent l'intégralité des coups
	possible, les notent et choisissent le meilleur ; cela en regardant plusieurs coups à l'avance.
	Voyons donc pourquoi avec ces techniques le jeu de Gygès est un jeu difficile.

	\subsection{Point de vue technique}

		\paragraph{Nombreux coups possible} Le grand nombre de coups évoqué ci-avant pose problème d'un point de vue temps de calcul.
		En effet, évaluer chaque coup en en regardant plusieurs en avance est très long. Hors pour que la machine soit efficace, il faut qu'elle
		soit capable d'anticiper plusieurs coups à l'avance.

		\paragraph{Coups semblables} Beaucoup de coups, notamment les remplacements, se ressemblent. Il est de plus possible d'accèder à un même plateau
		via plusieurs déplacements de pions différents. Il faudrait donc être capable de supprimer efficacement les doublons

		\paragraph{Cycle} Dans le Gygès, il y a beaucoup de cycle possible. Par exemple, il est possible de déplacer aucun pion au delà de la première
		ligne , voire de rester dans la même configuration que le plateau précedent. Enfin il est possible d'annuler le déplacement d'un pion de
		l'adversaire par un déplacement symètrique. Il faut donc savoir éviter ou détecter (pour les arrêter) ces cycles afin d'empêcher le jeu
		de tourner en rond.

		\paragraph{Chemins asymétrique} Les chemins construits sont différents selon les joueurs, cela veut dire
		que pour un plateau les deux n'ont pas les mêmes probabilités de gagner, il faut donc prendre en compte les
		deux sens d'un plateau lors de son évaluation.

	\vspace{2em}
	Mais le Gygès n'est pas seulement un jeu complexe pour les IA, en effet un joueur humain aussi
	cherchera à envisager les différents coups possible et à en choisir le meilleur.

	\subsection{Point de vue humain}

		\paragraph{Appartenance des pions} Les pions n'ont pas de propriétaire. Les deux joueurs ont la possibilité
		de déplacer chaque pion, cela implique de construire une stratégie ou chaque pièce pourra être déplacé.
		Par exemple, le pion que vous souhaitiez déplac au prochain tour pourra être déplacé alors qu'il se trouve
		sur votre ligne.

		\paragraph{Nombreux coups possible} \`A partir d'un plateau un joueur peut avoir accès à un très grand nombre de
		coups possible. Par exemple les possibilités d'ouverture d'une partie varie entre 150 et 600. Bien que parmi
		ces nombreuses possibilités beaucoup soient similaires, il reste difficile de trouver un coup utile n'ouvrant pas
		de chemin à l'adversaire.

		\paragraph{Complexité du jeu} Les déplacements complexes de pions et les parties rapides rendent difficile l'apprentissage
		du jeu pour les joueurs humains. De plus, le grand nombre de coup possible rend l'anticipation d'un coup plus difficile que
		dans les jeux traditionnels (échecs, dames).

		\paragraph{Chemins asymétrique} Les chemins construits par les joueurs ne sont pas forcément utilisable dans les deux sens.
		Il faut donc constamment s'assurer que les chemins construits ne donnent pas un avantage déterminant à l'adversaire.
