\section{Intérêt}

	La création d'intelligences artificielles (IA) de jeu présente plusieurs intérêts
	qui rendent leur utilisation utile pour la recherche en intelligence artificielle.

	\vspace{1em}
	Elles peuvent servir de benchmark, en effet un jeu permet de comparer
	de façon simple des IA entre elles ou contre des joueurs humain.
	Des facteurs tels que le nombre de coups nécessaire pour la victoire ou encore
	le ratio victoire/défaite sont des indicateurs de qualité facilement calculables
	et interprétables.

	\vspace{1em}
	De plus cette branche de l'intelligence artificielle fut beaucoup étudiée,
	il est donc aisé de trouver des algorithmes ou recherches sur lesquels se baser.
	Certains jeux furent l'objets de plus de travaux que d'autres, notament le jeu
	d'échecs qui fut le jeu par excellence pour la recherche en IA pendant plus de 40 ans,
	ou plus récemment le jeu de go.

	\begin{figure}[h]
		\centering
		\includegraphics[width=0.3\textwidth]{images/Deep_Blue.jpg}
		\caption{Deep blue: ordinateur ayant battu Kasparov aux échecs}
		\label{fig:deep_blue}
	\end{figure}
