%\begin{comment}
%	pour chaque fonction d'eval contre une IA aléatoire
%
%	1) nb moyen de noeud explorer selon prof
%	2) temps moyen de calcul selon profondeur
%	3) rapport 1)/2) selon prof
%	4) nombre de coups moyen pour gagner selon prof
%\end{comment}

\newpage
\section{Valeur des fonction d'evaluation}

	\subsection{Nombre moyen de n\oe{}uds explorés selon la profondeur}
		
		Un des points important d'une fonction d'évaluation est sa capacité
		à créer des valeurs qui vont être élagées, en effet moins de
		n\oe{}uds sont évalués plus rapide sera le choix du coup.
		Nous avons donc comparé le nombre de n\oe{}uds évalué en fonction
		de la profondeur pour chacune des fonctions d'évaluation.
		
		\begin{figure}[h!]
			\centering
			\includegraphics[width=\textwidth]{images/nbNodeDepth.png}
			\caption{N\oe{}uds/profondeur}
			\label{fig:stats_node_depth}
		\end{figure}
		
		On remarque qu'aux profondeurs 1 et 2 le faible nombre de n\oe{}uds
		évalués ne permet pas de différencier les différentes fonctions d'évaluation.
		\`A profondeur 3 et 4 la taille de l'arbre de recherche permet un plus grand
		élagage. Distance eval est la fonction qui permet le plus grand élagage,
		suivie par Maxpath eval. On découvre que Basic eval s'élague très mal, cela
		est dû au fait que les valeurs crées ne sont pas assez distinctes.
	
	\newpage
	\subsection{Temps moyen de calcul selon la profondeur}
	
		Un autre point important d'une fonction d'évaluation est sa vitesse,
		en effet on évalue jusqu'a plusieurs millions de n\oe{}uds, la vitesse
		est donc cruciale.
		Pour ce test nous avons mesuré le temps utilisé par la fonction sur une
		partie et nous l'avons divisé par le nombre de coups.
	
		\begin{figure}[h!]
			\centering
			\includegraphics[width=\textwidth]{images/timeDepth.png}
			\caption{Temps/profondeur}
			\label{fig:stats_time_depth}
		\end{figure}
		
		On remarque de nouveau que Distance eval est la fonction la plus
		rapide, notamment dû au faible nombre de n\oe{}uds évalués ainsi
		qu'à sa simplicitée.
		Pour Basic eval le temps est aussi directement lié au nombre de
		n\oe{}uds évalués.
		Enfin pour MaxPath eval on observe que malgré un nombre plus faible
		d'évaluations le temps requis à la profondeur 4 est plus grand,
		probablement dû à la complexité de la fonction.
	
	\newpage
	\subsection{Nombre de coup par milliseconde selon la profondeur}
	
		En faisant le ratio des deux premières statistiques, nous avons
		donc le nombre de n\oe{}uds évalués par milliseconde en fonction de la profondeur.
	
		\begin{figure}[h!]
			\centering
			\includegraphics[width=\textwidth]{images/ratio.png}
			\caption{Coups/profondeur}
			\label{fig:stats_ratio}
		\end{figure}
		
		On observe que MaxPath eval a un nombre assez constant
		quelque soit la profondeur, ce qui n'est pas le cas de
		Basic eval et Distance eval, mais leurs valeurs n'oscillent
		que entre 20 et 100, ce n'est donc pas vraiment influent
		sur la capacité des fonctions.
	
	\newpage
	\subsection{Nombre moyen de coups pour gagner selon la profondeur}
	
		Au final le point le plus important d'une fonction d'évaluation est
		sa capacité à gagner. Nous avons donc fait combatre les fonctions
		d'évaluation contre une fonction jouant au hasard et nous avons
		regarder le nombre de coups nécessaire pour gagner.
	
		\begin{figure}[h!]
			\centering
			\includegraphics[width=\textwidth]{images/victNbMoves.png}
			\caption{Coups/profondeur}
			\label{fig:stats_nbMovesVict}
		\end{figure}
		
		On note que MaxPath eval qui est la fonction la plus complexe
		est aussi la fonction la plus performante, nécéssitant que de 2 à 4
		coups pour gagner. A partir de la profondeur 3 Basic eval gagne aussi
		rapidement. Mais Distance eval, la fonction la plus simple et la plus
		rapide, a les scores les plus mauvais, nécéssitant plus de 5 coups
		pour gagner.
