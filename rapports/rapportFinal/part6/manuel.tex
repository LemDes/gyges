\chapter{Manuel}

	\section{Ligne de commande}
	
		On utilise le logiciel en utilisant la commande
		\texttt{java -jar gyges.jar}, cette commande prend
		plusieurs paramètres pour régler la partie.
		
		\vspace{1em}
		On peut choisir la fonction d'évaluation utilisée avec les
		paramètres \texttt{-{}-eval1 NomDeLaClasse} et \texttt{-{}-eval2 NomDeLaClasse},
		permettant de changer la fonction pour le joueur 1 et le joueur 2.
		Sont disponible les fonctions suivante: \emph{BasicEval}, \emph{DistanceEval} et
		\emph{MaxPathEval}.
		
		C'est la fonction basic eval qui est utilisé par defaut si aucune fonction
		n'est passée en paramètes ou si elle est incorrecte.
		
		\vspace{1em}
		De même on peut choisir l'algorithme de recherche avec les paramètres \texttt{-{}-ai1 NomDeLaClasse}
		et \texttt{-{}-ai2 NomDeLaClasse}.
		Sont disponible les algorithmes suivant: \emph{NegamaxMover} et \emph{MtdfMover}.
		
		C'est l'algorithme MTD-f qui est utilisé par defaut.
		
		\vspace{1em}
		Ces algorithme prennent en paramètre la profondeur de recherche qui peuvent
		être configurés avec les paramètres \texttt{-{}-depthJ1 Profondeur} et\\
		\texttt{-{}-depthJ2 Profondeur} ou avec \texttt{-{}-depth Profondeur} pour régler les deux
		en même temps.
		
		La valeur par défaut est de 2.
		
		\vspace{1em}
		Il est possible de définir les positions de départ avec les paramètres
		\texttt{--pos1 Position} et \texttt{--pos2 Position} où \emph{Position}
		est une suite de six chiffres réprésentant les pions.
		
	\section{Interface graphique}
		On peut lancer une interface graphique avec l'option \texttt{-{}-ui}, dans ce cas
		on peut rajouter des joueurs humains en passant la valeur \texttt{human} à un ou
		aux deux paramètres \texttt{-{}-ai}.
		La création de joueurs humains passent forcement le jeu en mode interface graphique.
		
		\vspace{1em}
		Le joueur humain sélectionne un pion avec un clic droit de souris, effectue
		un rebond avec un clic de molette et pose le pion avec le clic gauche.
		
		\begin{figure}[h!]
			\centering
			\includegraphics[width=\textwidth]{images/uiPlaying.png}
			\caption{Capture d'écran du jeu}
			\label{fig:screenshot_game}
		\end{figure}
