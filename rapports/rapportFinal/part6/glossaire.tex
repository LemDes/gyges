\chapter{Glossaire}

	\begin{description}[style=nextline]
		%\item[Algorithme Search EXtension] Une amélioration du \emph{minimax} est le \emph{Search EXtension},
		%cet algorithme permet d'avoir une profondeur de recherche variable. Plus le coup semble intéressant, plus
		%la branche sera explorée et au contraire, moins le mouvement est intéressant, moins on explorera la branche.

		\item[Chemin] Ensemble de pions permettant de se déplacer via rebond.

		\item[Cycle] Ensemble de coups remettant le jeu dans sa position précédente.

		\item[\'Elagage Alpha-Beta] Algorithme visant à réduire le nombre de n\oe{}uds
		évalués par un algorithme minimax en excluant les n\oe{}uds qui ne contriburont
		pas au calcul du gain de la racine.

		\item[Intelligence artificielle] Système informatique mimant les capacités de réflexions
		d'un être humain. Ici fait référence à la partie du programme émulant un joueur.

		\item[Facteur de branchement] Nombre de plateaux possible depuis un plateau départ.

		\item[Fonction d'évaluation] Fonction prenant en entrée un plateau de jeu et sortant la valeur de
		ce plateau.

		\item[Heuristique] Algorithme qui fourni une approximation d'un résultat de façon plus rapide
		qu'un algorithme exacte.

		\item[Horizon] Limite de calcul dans l'arbre de recherche. Fait référence au fait qu'un très bon coup peut
		se cacher derrière le dernier coup calculé, mais qu'il est impossible de le savoir.

		\item[Jeu combinatoire abstrait] Type de jeu où deux joueurs ou équipes d'affrontent et jouent à tour de rôle,
		tous les éléments sont connus et où le hasard n'intervient pas.

		\item[Minimax] Simple, cet algorithme assigne à un n\oe{}ud la valeur minimale ou maximale de ses
		n\oe{}uds fils, respectivement lorsque le n\oe{}ud resprésente le tour de l'adversaire ou le tour
		du joueur. Pour un n\oe{}ud feuille sa valeur correspond à la valeur retournée par la fonction
		d'évaluation pour le plateau représenté par ce n\oe{}ud. Cet algorithme cherche donc à minimiser
		les pertes tout en maximisant les gains.

		\item[MTD-f] Une optimisation du \emph{minimax}, le \emph{MTD-f} se sert de tables de transposition
		pour garder en mémoire la valeur d'un plateau et ainsi ne noter qu'une seule fois les plateaux identiques.
		C'est une des variantes du \emph{minimax} la plus rapide.

		\item[Negamax] Une variante du \emph{minimax}, en se basant sur le fait
		que $max(a,b) = -min(-a,-b)$ et que la valeur de coup pour le joueur adverse et la négation du
		coup pour le joueur courant, pour simplifier l'algorithme en n'utilisant que des \emph{max}
		au lieu d'une alternance \emph{max}, \emph{min}.

		\item[Neurone] Un neurone artificiel est composé d'entrées pondérées, plus le poids est fort plus l'entrée
		influence la valeur de sortie. Les valeurs des entrées sont combinés grâce à une fonction de
		combinaison, le plus souvent la fonction somme suivante: $net_j = \sum\limits_{i=0}^n x_i  w_{ij}$.
		Enfin la valeur de sortie du neurone est calculée de la façon suivante: $O_j = \tanh{}(net_j)$.
		L'utilisation de la tangente hyperbolique, une fonction Sigmoïde, permet une activation non linéaire
		et possède une sortie dans l'intervale -1 à 1.

		\item[Perceptron multicouche] Réseau de neurones formel organisé en plusieurs couche utilisant une fonction de sortie
		de type sigmoïde. Utiliser comme classifieur non linéaire ou pour la régression de fonction inconnue.

		%\item[Recherche Monte-Carlo (Monte-Carlo Tree Search)] La recherche arborescente de Monte-Carlo est une recherche
		%où l'on construit un arbre de recherche n\oe{}uds à n\oe{}uds en fonction du résultat de parties simulées aléatoirement.

		\item[Table de transposition] Permet de conserver en mémoire l'évaluation d'un n\oe{}ud et ainsi éviter une réévalutation
		coûteuse.
	\end{description}
