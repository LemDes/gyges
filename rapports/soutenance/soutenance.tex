\documentclass{beamer}

\usepackage[francais]{babel}
\usepackage[utf8]{inputenc}
\usepackage[T1]{fontenc}

\usepackage{beamerthemesplit}
\usetheme{Boadilla}
\usecolortheme{beaver}
\setbeamertemplate{navigation symbols}{}
\useinnertheme{circles}

\title[Intelligence artificielle pour le Gygès]
{%
	Implémentation d’algorithmes d’intelligence artificielle pour le Gygès
}
\author{Valentin~\textsc{Lemière} \and Guillaume~\textsc{Desquesnes}}
\date{26 mars 2013}

\AtBeginSection[]
{
	\begin{frame}{Plan}
		\begin{columns}
			\column{0.2\textwidth}
			\column{0.8\textwidth}
				\tableofcontents[currentsection]
		\end{columns}
	\end{frame}
}

\begin{document}

	\frame{\titlepage}

	\section{Présentation}

	\begin{frame}{Le but}
		\begin{block}{But du projet}
			\vspace{0.75em}
			Réaliser une \alert{intelligence artificielle} de Gygès.
			\vspace{0.75em}
		\end{block}
	\end{frame}

	\begin{frame}{Qu’est ce que le Gygès ?}
		\begin{columns}
			\column{0.5\textwidth}
				Le Gygès est :
				\vspace{1em}
				\begin{itemize}
					\item<2-> Un jeu
					\vspace{0.75em}
					\item<3-> \`A deux joueurs
					\vspace{0.75em}
					\item<4-> Un plateau $6\times{}6$ et 12 pions
					\vspace{0.75em}
					\item<5-> Le but: placer un pion dans la base adverse.
				\end{itemize}

			\column{0.5\textwidth}
				\alt<4> {
					\begin{figure}[h]
						\centering
						\includegraphics[width=\textwidth]{images/Gyges.png}
					\end{figure}
				} {
					\alt<5> {
						\begin{figure}[h]
							\centering
							\includegraphics[width=\textwidth]{images/Bases.png}
						\end{figure}
					} {
						\uncover<4> {
							\begin{figure}[h]
								\centering
								\includegraphics[width=\textwidth]{images/Bases.png}
							\end{figure}
						}
					}
				}
		\end{columns}
	\end{frame}

	\begin{frame}{Qu’est ce que le Gygès ?}
		\begin{columns}
			\column{0.5\textwidth}
				Les règles du Gygès:

				\vspace{1em}
				\begin{itemize}
					\item<2-> les pions
					\vspace{0.75em}
					\item<3-> leurs déplacements
					\vspace{0.75em}
					\item<6-> des déplacements spéciaux:
					\begin{itemize}
						\item<7-> les rebonds
						\vspace{0.75em}
						\item<8-> les remplacements
					\end{itemize}
				\end{itemize}

			\column{0.5\textwidth}
				\alt<2> {
					\begin{figure}[h!]
						\centering
						\includegraphics[width=\textwidth]{images/pawns.png}
					\end{figure}
				} {
					\alt<3-5> {
						\only<3> {
							\begin{figure}[h!]
								\centering
								\includegraphics[width=\textwidth]{images/move1.png}
							\end{figure}
						}

						\only<4> {
							\begin{figure}[h!]
								\centering
								\includegraphics[width=\textwidth]{images/move2.png}
							\end{figure}
						}

						\only<5> {
							\begin{figure}[h!]
								\centering
								\includegraphics[width=\textwidth]{images/move3.png}
							\end{figure}
						}
					} {
						\alt<7> {
							\begin{figure}[h!]
								\centering
								\includegraphics[width=\textwidth]{images/rebond.png}
							\end{figure}
						} {
							\alt<8> {
								\begin{figure}[h!]
									\centering
									\includegraphics[width=\textwidth]{images/Rempl.png}
								\end{figure}
							} {
								\uncover<2> {
									\begin{figure}[h]
										\centering
										\includegraphics[width=\textwidth]{images/Gyges.png}
									\end{figure}
								}
							}
						}
					}
				}
		\end{columns}
	\end{frame}

	\begin{frame}{Pourquoi?}
		\begin{columns}
			\column{0.5\textwidth}
				\begin{itemize}
					\item<1-> Les jeux sont fréquement utilisés comme support de recherche.
					\vspace{0.75em}
					\item<2-> Les échecs furent extrémement étudiés.
					\vspace{0.75em}
					\item<3-> Le Gygès a un \alert{facteur de branchement} très élevé, plus que les échecs.
					\vspace{0.75em}
					\item<4-> Le Gygès présente donc un intérêt de modélisation.
				\end{itemize}

			\column{0.1\textwidth}
			\column{0.4\textwidth}
				\uncover<3-> {
					\begin{alertblock}{Facteur de branchement}
					Nombre de coups possibles depuis un plateau.
					\end{alertblock}
				}
		\end{columns}
	\end{frame}

	\begin{frame}{Les difficultés} % Keep ?
		\begin{columns}
			\column{0.5\textwidth}
				Le Gygès est un jeu complexe, il a en effet:

				\vspace{1em}
				\begin{itemize}
					\pause \item Un fort facteur de branchement
					\vspace{0.75em}
					\pause \item Des coups qui se ressemblent% | pas facile de les différencier | déterminer si c'est un bon coup
					\vspace{0.75em}
					\pause \item Des chemins asymétriques
				\end{itemize}

			\column{0.5\textwidth}
				\uncover<3-> {
					\begin{alertblock}{Chemin asymétrique}
						Un chemin de pions qui ne peut être pris
						que dans un seul sens.
					\end{alertblock}
				}
		\end{columns}
	\end{frame}

	\begin{frame}{Objectifs}
		Nous avions comme objectifs de développer une IA\\
		de jeu de Gygès capable de:

		\vspace{1em}
		\begin{itemize}
			\pause \item déterminer quels pions peuvent être joués
			\vspace{0.75em}
			\pause \item lister les coups possibles
			\vspace{0.75em}
			\pause \item évaluer un coup
			\vspace{0.75em}
			\pause \item déterminer un bon coup en regardant plusieurs coups à l'avance
			\vspace{0.75em}
			\pause \item jouer un coup
		\end{itemize}
	\end{frame}

	\section{Développement}

	\begin{frame}{Comment déterminer un coup ?}
		\centering

		Pour déterminer un coup :
		\pause $$\downarrow$$
		on liste tout les coups possibles\pause{},
		$$\downarrow$$
		on leur assigne une note\pause{},
		$$\downarrow$$
		on choisit le meilleur.
	\end{frame}

	\begin{frame}{Les fonctions d'évaluation}

		Pour attribuer une note à un plateau on utilise une
		\alert{fonction d'évaluation}.

		\vspace{1em}
		\begin{block}{Fonction d'évaluation}
		\vspace{0.75em}
			Une fonction qui attribue une note à un
			plateau, elle représente une stratégie,
			une manière de jouer.
		\vspace{0.75em}
		\end{block}
		\vspace{1em}

		%Cette fonction retourne un entier correspondant
		%à la valeur du plateau.

	\end{frame}

	\begin{frame}{Les fonctions d'évaluation}
		Nous avons trois fonctions d'évaluation différentes:

		\vspace{1em}
		\begin{itemize}
			\pause \item BasicEval
			\vspace{0.75em}
			\pause \item DistanceEval
			\vspace{0.75em}
			\pause \item MaxPathEval
		\end{itemize}
	\end{frame}

	\begin{frame}{Choisir un bon coup}
		\begin{columns}
			\column{0.5\textwidth}
				\begin{itemize}
					\item<1-> On liste les coups sous forme d'arbre.
					\vspace{0.75em}
					\item<2-> On leur assigne une note en utilisant une fonction d'évaluation.
					\vspace{0.75em}
					\item<3-> Minimax
					\vspace{0.75em}
					\item<5-> On choisit le meilleur.
				\end{itemize}

			\column{0.5\textwidth}
				\alt<1> {
					\begin{figure}[h]
						\centering
						\includegraphics[width=\textwidth]{images/arbre_recherche.png}
					\end{figure}
				} {
					\alt<2> {
						\begin{figure}[h]
							\centering
							\includegraphics[width=\textwidth]{images/minimax_ex_1.png}
						\end{figure}
					} {
						\alt<3> {
							\begin{figure}[h]
								\centering
								\includegraphics[width=\textwidth]{images/minimax_ex_2.png}
							\end{figure}
						} {
							\begin{figure}[h]
								\centering
								\includegraphics[width=\textwidth]{images/minimax_ex_3.png}
							\end{figure}
						}
					}
				}

		\end{columns}
	\end{frame}

	\begin{frame}{Choisir un bon coup}
		\begin{columns}
			\column{0.5\textwidth}
				Le MTD-f:

				\vspace{1em}
				\uncover<1-> {
					Utilise des tables de transposition
				}

				\vspace{1em}
				\uncover<2-> {
					Pour faire des recherches plus complexes.
				}

			\column{0.5\textwidth}
				\uncover<1-> {
					\begin{alertblock}{Table de transposition}
						Mémorise la valeur de l'évaluation d'un coup.
					\end{alertblock}
				}
		\end{columns}
	\end{frame}

	\section{Bilan}

	\begin{frame}{Bilan - Fonction d'évaluations}
		\begin{figure}[h!]
			\centering
			\includegraphics[width=0.7\textwidth]{images/nbNodeDepth.png}
		\end{figure}
	\end{frame}

	\begin{frame}{Bilan - Fonction d'évaluations}
		\begin{figure}[h!]
			\centering
			\includegraphics[width=0.7\textwidth]{images/victoireProfFixe.jpg}
		\end{figure}
	\end{frame}

	\begin{frame}{Bilan}
		Perspectives du projet:

		\vspace{1em}
		\begin{itemize}
			\item Ajout de nouveaux algorithmes de recherche.
			\vspace{0.75em}
			\item Ajout de meilleures fonctions d'évaluation.
			\vspace{0.75em}
			\item Améliorer interface graphique.
		\end{itemize}
	\end{frame}

	\section{Démonstration}

	\begin{frame}{Démonstration}
		Démo
	\end{frame}

\end{document}
