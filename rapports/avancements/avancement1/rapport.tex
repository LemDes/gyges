\documentclass[a4paper]{article}

\usepackage[T1]{fontenc}
\usepackage[utf8]{inputenc}
\usepackage{lmodern}
\usepackage[french]{babel}

\title{Rapport d'avancement du projet de Gygès}
\author{Valentin \textsc{Lemière} - Guillaume \textsc{Desquesnes}}
\date{}

\begin{document}
\maketitle

\section*{Synthèse}
	Le projet avance normalement, l'objectif du jalon ayant été atteint plus tôt nous 
	avons pu implémenter une fonction supplémentaire permettant de tester l'IA.
	Nous avons trouvé une alternative au negamax, le MTD-f.

\section*{Avancement}
	\subsection*{Achèvements sur la période}
		\begin{itemize}
			\item Nouvelle fonction d'évaluation d'un plateau ;
			\item Possibilité à un joueur d'affronter l'IA ;
			\item Fonction de recherche du meilleur coup prenant en compte plusieurs coups à l'avance ;
			\item Recherche sur des alternatives au negamax.
		\end{itemize}		
		
	\subsection*{En cours sur la période}
		\begin{itemize}
			\item Modification de la fonction de test d'IA pour permettre l'affrontement de deux IA différentes ;
			\item Tests de la fonction d'évaluation pour avoir un point de comparaison
			avec les futures fonctions.			
		\end{itemize}
	
	\subsection*{A venir sur la période suivante}
		\begin{itemize}
			\item Recherche sur les tables de transposition, pour le MTD-f ;
			\item \'Etude de complexité des fonctions d'évaluations ;
			\item Implémentation d'une fonction plus avancée pour augmenter la profondeur de recherche.
		\end{itemize}
	
\section*{Problèmes}		
	\subsection*{Problèmes apparus et résolus au cours de la période}
		\begin{itemize}
			\item Résolution d'un problème de disparition de pions du à une mauvaise détection de victoire.
		\end{itemize}

\section*{Actions}
	\begin{itemize}
		\item Optimisation du negamax: recherche aspirante ;
		\item \'Etude de complexité des fonctions d'évaluation ;
		\item Recherche sur les tables de transpositions ;
		\item Faire une fonction d'évaluation à la complexité algorithmique plus faible.
	\end{itemize}

\end{document}
